\documentclass{article}
\usepackage[utf8]{inputenc}
\usepackage[margin=0.5in]{geometry}
\usepackage{lipsum}

\usepackage{jlcode}
\usepackage{multicol}
\usepackage{listings}

\title{Julia Cheatsheet}
\author{Brian Jackson}

\begin{document}
\maketitle
\begin{multicols*}{3}
\section{Notation}
\texttt{a,b,c} are scalars,
\texttt{x,y,z} are vectors, and
\texttt{A,B,C} are matrices.
\texttt{S} is a square matrix,
\texttt{s} is a string.
\texttt{elw} means element-wise.

\section{Assignment}
\begin{verbatim}
a = 1   # scalar assignment
A = B   # alias assignment
A .= B  # element-wise copy
A = copy(B)      # copy
A = deepcopy(B)  # deep copy
a = 1 + 2im  # imaginary
b = 1 // 2   # rational
\end{verbatim}

\section{Scalar Arithmetic}
\begin{verbatim}
a+2 # addition 
a-1 # subtraction
2*a # multiplication
2a  # multiplication
a/3 # float division
a÷3 # int division (\div)
a^2 # exponential
a%3 # modulus
\end{verbatim}

\section{Arrays}
\subsection{Initialization}
\vspace*{-2mm}
\begin{verbatim}
[1,2,3]       # vector
[1 2 3]       # row vector
[1; 2; 3]     # col vector
[1 2; 3 4]    # 2x2 matrix
zeros(3)      # all 0s (vec)
zeros(3,2)    # all 0s (mat)
ones(3,2)     # all 1s 
ones(Int,3,2) # Integer 1s
rand(3,2)  # uniform from 0-1
randn(3,2) # Std. Gaussian 
fill(10,3,2)  # all 10s
0:10     # integer range 0-10
0:2:10   # range 0,2,4,...,10
range(0,10,step=2) # same
0:0.1:10 # step of 0.1
range(0,10,length=101) # same
\end{verbatim}

\subsection{Arithmetic}
\vspace*{-2mm}
\begin{verbatim}
2 .+ x  # scalar add
2 .- x  # scalar sub 
x + y   # elw add
x - y   # elw sub
x .* y  # elw mult
x ./ y  # elw div 
x' * y  # dot product
x'y     # dot product
x * y'  # outer product
x * y   # undefined
A * B   # matrix mult
A .* B  # elw mult
A .^ 2  # elw square
S^2     # matrix mult 
\end{verbatim}

\subsection{Indexing}
\vspace*{-2mm}
Assume \texttt{size(A) == (3,2)}.
\vspace*{-2mm}
\begin{verbatim}
x[1]   # linear index
A[4]   # linear index
A[1,2] # row,col (same)
x[2:end]   # 2nd to last
x[1:end-2] # 1st to 3rd last
A[:,1]     # 1st column
A[1:2,:]   # first 2 rows
A[1] = 2   # assign element
A[:,1] .= 2  # assign range
A[:,1] = x   # assign range
\end{verbatim}

\section{Other Types}
\subsection{Strings}
\vspace*{-2mm}
\begin{verbatim}
'c'          # char
"my string"  # string
:abc         # symbol (fast)
"my num: $a" # interpolation
string("a","b") # concat
"a" * "b1"      # concat
s[1]            # get char
s[1:2]          # sub-string
\end{verbatim}

\subsection{Dictionaries}
\vspace{-2mm}
\begin{verbatim} 
d1 = Dict(:a=>1, :b=>2)
d2 = Dict("d"=>x, "e"=>y)
d1[:a]         # indexing
d2["g"] = x+y  # new entry 
pop!(d2, "g)   # remove entry
keys(d1)       # get keys
values(d1)     # get values
for (k,v) in pairs(d1) 
    # key k, value v 
end
\end{verbatim}

\subsection{Lists}
\vspace{-2mm}
\begin{verbatim}
[1,2,3]        # good
["a","b","c"]  # good
[[1,2],[2]]    # good
[1,"b",[1,2]]  # avoid 
maximum(x) # maximum element
minimum(x) # minimum element
argmax(x)  # index of max
findmax(x) # (val,idx) of max
push!(x,1) # add to end
insert!(x,1,5) # add to start
append!(x,y) # concat 
[x; y]       # vert cat
[x y]        # horz cat
vcat(x,y)    # vert cat
hcat(x,y)    # horz cat
a in x       # exists in?
sort(x)      # sort
sort!(x)     # sort in-place
sortperm(x)  # sort indices
\end{verbatim}


\subsection{Other}
\vspace{-2mm}
\begin{verbatim}
(1,2,3)       # tuple
Set((1,2,3))  # set
\end{verbatim}
    
\section{Control Flow}
\subsection{Logic}
\vspace*{-2mm}
\begin{verbatim}
a == b  # are equal
A == B  # all elm are equal
isapprox(a, b) # \approx
A === B # same memory loc
a != b  # not equal
a < b   # less than
a <= b  # less than or equal
a && b  # short-circuit and
a || b  # short-circuit or 
a < b < c # b between a,c
\end{verbatim}

\subsection{Conditionals}
\vspace*{-2mm}
\begin{verbatim}
if a < b
    # code
elseif b > a
    # code
else
    # code
end
a < b ? 1 : 0  # inline
(a < b) && 1   # short-circuit
\end{verbatim}

\subsection{Loops}
\vspace*{-2mm}
\begin{verbatim}
# For loops
for x = 1:10
    # loop body
end
for a in x  # or \in
    # loop body
end
for i = 1:10, j = 1:10
    # nested loop
end

# While Loop
while (a < b)
    a += 1
end

# List comprehension
x = [sin(i) for i = 1:10]
A = [i+j for i in x, j in y]
\end{verbatim}

\subsection{Functions}
\vspace*{-2mm}
\begin{verbatim}
function myfun(x,y,a=1;b=2)
    # function body
    return <expression>
end
# valid calls
myfun(1,2)
myfun(1,2,3)
myfun(1,2,3,b=3)
myfun(1,2,b=3)

# anonymous functions
mysum(x,y) = x+y
mysub = (x,y) -> x-y
\end{verbatim}


\section{Linear Algebra}
\vspace*{-2mm}
\begin{verbatim}
using LinearAlgebra
norm(x)     # 2 norm
norm(x,Inf) # Inf norm
norm(x,p)   # p-norm
diag(A)     # get diagonal
inv(S)      # inverse
eigvals(S)  # eigenvalues
rank(S)     # rank
cond(S)     # condition num 
isposdef(S) # x'S*x > 0?
Diagonal(x)  # diag mat
Symmetric(S) # symm mat
y = A\x      # solve Ax = y
eigen(S)     # Eigen decomp 
qr(S)        # QR fact
svd(S)       # SVD fact
cholesky(S)  # Cholesky
\end{verbatim}

\section{Useful Macros}
\subsection{Benchmarking}
\vspace*{-2mm}
\begin{verbatim}
@time f(x)      # print time
@elapsed f(x)   # get time
@allocated f(x) # get allocs

# to run many times
using BenchmarkTools
@btime f(x)     # print time
@benchmark f(x) # get details
\end{verbatim}

\subsection{Other}
\vspace*{-2mm}
\begin{verbatim}
# get which method is called
@which f(x)

# type stability info
@code_warntype f(x)
\end{verbatim}

\section{Packages}
\vspace*{-2mm}
\begin{verbatim}
# load package to use
using MyPackage 

# Don't import methods
import MyPackage

# shorten name
const MP = MyPackage

# load specific methods
using MyPackage: foo, bar 

# load methods to redefine 
import MyPackage.foo
\end{verbatim}

\subsection{Adding/Removing}
In REPL, type \texttt{]} to open 
package manager. Here \texttt{Pack}
can be any package name.
\begin{verbatim}
add Pack        # add
add Pack@1      # add version
add Pack#master # add branch
rm Pack         # remove
activate dir  # use env at dir
st  # list installed packages
\end{verbatim}

% \section{Type System}
% \subsection{Basic Ops}
% \vspace*{-2mm}
% \begin{verbatim}
% typeof(a) # Float64
% typeof(x) # Array{Float64,1}
% x isa Vector{Float64} # true
% Vector <: Array # true 
% Int <: Number   # true
% \end{verbatim}

% \subsection{Custom Types}
% \vspace*{-2mm}
% \begin{verbatim}
% abstract type Phasors end
% struct P1 <: Phasors
%     a::Float64
%     b::Float64
%     isnorm::Bool
% end # fields can't be changed
% mutable struct P2 <: Phasors
%     a::Float64
%     b::Float64
%     isnorm::Bool
% end # fields can be changed
% function foo(x::Phasors)
%     # define foo on both
%     x.a + x.b  # return this
% end
% \end{verbatim}


\end{multicols*}

\end{document}